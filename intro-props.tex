% !TEX root = laws.tex

\newcommand{\avector}[2]{(#1_1,#1_2,\ldots,#1_{#2})}
\newcommand{\aDEFvector}[2][a]{(#1_1,#1_2,\ldots,#1_{#2})}

\subsection{Property-based testing}
In the scope of a data domain $\mathbb{D}$, a property can be seen as a collective abstract behaviour which has to be followed by every valid member of that data domain. In other words, a property can be understood as a predicate $P$ on a variable $X$ ($P:X \rightarrow \{true, false\}$) such that: 
\begin{center}
$\forall X \in \mathbb{D}, P(X) = true$
\end{center}
Consider an example of a property $P$ that has to be satisfied by the domain of all strings $\mathbb{S}$.
\begin{center}
$\forall X \in \{s_1::s_2 | \#s_1::s_2 >  \#s_1 \wedge s_1, s_2 \in \mathbb{S}\}, P(X) = true$
\end{center}
where $::$ denotes a regular string append operation. But if we look closely, we observe that if $s_1, s_2 \in \{\phi\}$ then the property $P$ does not hold true anymore contrary to the fact that both component strings are valid members of $S$. Hence, $P$ is not a valid property over the domain $S$. \\
In contrast to conventional testing methods where it might suffice to test the behaviour of some boundary points at the discrete neighbourhoods, property-based testing emphasises on defining universal (within the domain) properties and then testing their validity against randomly sampled data points. There are some popular libraries available for property testing like QuickCheck for Haskell,  and ScalaTest and ScalaCheck (majorly for generator-driven property testing) for Scala-based programs. \\
Since it might not be the case everytime that the random data points needed for property testing are indeed the primitive data types, so in such cases generators are defined, which are simple functions that usually generate some output based on input parameters, to facilitate data generation for testing. A simple example of a generator $G$ can be:
\begin{center}
$G(X) \rightarrow {Y}$, where $X=\aDEFvector[x]{n+1}$ and $Y=\{x | x \% 2 = 0\}$
\end{center}
We now show an example of a typical property and a generator in context of ScalaCheck [Add reference here]. 