\documentclass[conference]{IEEEtran}   % list options between brackets

\usepackage{color}
\usepackage{graphicx}
%% The amssymb package provides various useful mathematical symbols
\usepackage{amssymb}
%% The amsthm package provides extended theorem environments
%\usepackage{amsthm}
\usepackage{amsmath}

\usepackage{listings}

\usepackage{hyperref}

\usepackage{systeme}

\def\shownotes{1}
\def\notesinmargins{0}

\ifnum\shownotes=1
\ifnum\notesinmargins=1
\newcommand{\authnote}[2]{\marginpar{\parbox{\marginparwidth}{\tiny %
  \textsf{#1 {\textcolor{blue}{notes: #2}}}}}%
  \textcolor{blue}{\textbf{\dag}}}
\else
\newcommand{\authnote}[2]{
  \textsf{#1 \textcolor{blue}{: #2}}}
\fi
\else
\newcommand{\authnote}[2]{}
\fi

\newcommand{\knote}[1]{{\authnote{\textcolor{green}{Alex notes}}{#1}}}
\newcommand{\mnote}[1]{{\authnote{\textcolor{red}{Mayank notes}}{#1}}}

\usepackage[dvipsnames]{xcolor}
\usepackage[colorinlistoftodos,prependcaption,textsize=tiny]{todonotes}

%\newcommand{\bruno}[1]{\todo[linecolor=OliveGreen,backgroundcolor=OliveGreen!25,bordercolor=OliveGreen]{#1}}

% type user-defined commands here
\usepackage[T1]{fontenc}

%\usepackage{xcolor}


\newcommand{\ma}{\mathcal{A}}
\newcommand{\mb}{\mathcal{B}}
\newcommand{\he}{\hat{e}}
\newcommand{\sr}{\stackrel}
\newcommand{\ra}{\rightarrow}
\newcommand{\la}{\leftarrow}
\newcommand{\state}{state}

\newcommand{\ignore}[1]{} 
\newcommand{\full}[1]{}
\newcommand{\notfull}[1]{#1}
\newcommand{\rand}{\stackrel{R}{\leftarrow}}
\newcommand{\mypar}[1]{\smallskip\noindent\textbf{#1.}}

\begin{document}

\title{Checking Laws of the Blockchain With Property-Based Testing}
\author{\IEEEauthorblockN{Alexander Chepurnoy\IEEEauthorrefmark{1}, Mayank Rathee\IEEEauthorrefmark{2}}
\IEEEauthorblockA{\IEEEauthorrefmark{1} Ergo Platform and IOHK Research\\
Sestroretsk, Russia}
\IEEEauthorblockA{\IEEEauthorrefmark{2} Department of Computer Science and Engineering, Indian Institute of Technology (Banaras Hindu University)\\
Varanasi, India}}

\maketitle

\begin{abstract}
Inspired by the success of Bitcoin, many clients for the Bitcoin protocol as well as for alternative blockchain protocols have been implemented. However, implementations may contain errors, and the cost of an error in the case of a cryptocurrency can be extremely high. 

We propose to tackle this problem with a suite of abstract property tests that check whether a blockchain system satisfies laws that most blockchain and blockchain-like systems should satisfy. To test a new blockchain system, its developers need to instantiate generators of random objects to be used by the tests. The test suite then checks the satisfaction of the laws over many random cases. We provide examples of laws in the paper.
\end{abstract}

% !TEX root = laws.tex


\section{Introduction}
% !TEX root = laws.tex

\newcommand{\avector}[2]{(#1_1,#1_2,\ldots,#1_{#2})}
\newcommand{\aDEFvector}[2][a]{(#1_1,#1_2,\ldots,#1_{#2})}

\subsection{Property-based testing}
In this section, we give a formal definition of a property followed by a discussion on property-based testing in contrast to the conventional testing methodologies.\\
Within the scope of a data domain $\mathbb{D}$, a property can be seen as a collective abstract behaviour which has to be followed by every valid member of the data domain. In other words, a property can be understood as a predicate $P$ over a variable $X$ ($P:X \rightarrow \{true, false\}$) such that: 
\begin{center}
$\forall X \in \mathbb{D}, P(X) = true$
\end{center}
Let us consider an example of a property $P$ over the domain of all possible strings $\mathbb{S}$.
\begin{center}
$\forall X \in \{s_1::s_2 | \#s_1::s_2 >  \#s_1 \wedge s_1, s_2 \in \mathbb{S}\}, P(X) = true$
\end{center}
where $::$ denotes a regular string append operation and $\#s$ denotes the length of string $s$. But if we look closely, we observe that if $s_1, s_2 \in \{\phi\}$ then the property $P$ does not hold true anymore contrary to the fact that both component strings are valid members of $\mathbb{S}$. Hence, $P$ is not a valid property over the domain $\mathbb{S}$. \\
In contrast to conventional testing methods where it might suffice to test the behaviour of some boundary points at the discrete neighbourhoods, property-based testing emphasises on defining universal (within the domain) properties and then testing their validity against randomly sampled data points. There are some popular libraries available for property testing including QuickCheck for Haskell \cite{claessen2011quickcheck}, JUnit-QuickCheck for Java, theft for C, ScalaTest and ScalaCheck \cite{nilsson2014scalacheck} (majorly for generator-driven property testing) for Scala-based programs.
\knote{improve the bridge}
Walking inside the walls of bitcoin and other cryptocurrencies, we, in this work, highlight some of these properties which should hold true for any implementation of a valid cryptocurrency.
% !TEX root = laws.tex

\subsection{Scorex Framework}

The idea of a modular design for a cryptocurrency was first proposed by Goodman in the Tezos position paper~\cite{tezosPosition}. The paper~\footnote{Section 2 of the paper} proposes to split a cryptocurrency design into the three protocols: network, transaction and consensus. In many cases, however, these layers are tightly coupled and it is hard to describe them separately. For example, in a proof-of-stake cryptocurrency a balance sheet, which representation is heavily influenced by a transaction format, is used in a consensus protocol. 

Plenty of modular open-source frameworks were proposed for speeding up development of new blockchains: Sawtooth~\cite{sawtooth} and Fabric~\cite{fabric} by Hyperledger, Exonum~\cite{exonum} by Bitfury Group, Scorex~\cite{scorex} by IOHK etc. We have chosen Scorex, as it has finer granularity. In particular, in order to support hybrid blockchains as well as more complicated linking structures than a chain~(such as Spectre\cite{spectre}), Scorex does not have a notion of a blockchain as a core abstraction. Instead, it provides an abstract interface to a \textit{history} which contains \textit{persistent modifiers}. The history is a part of a \textit{node view}, which is a quadruple of $\langle$\textit{history}, \textit{minimal state}, \textit{vault}, \textit{memory pool}$\rangle$. This node view is to be updated as a result of processing whether a persistent modifier or a transaction. The minimal state is a data structure and a corresponding interface providing the ability to check the validity of an arbitrary persistent modifier for the current moment of time with the same result for all the nodes in the network having the same history. The minimal state is to be obtained deterministically from an initial pre-historical state and the history. The vault holds node-specific information, for example, a node user's wallet. The memory pool holds unconfirmed transactions being propagated across the networks by nodes before their inclusion into blocks. Such a design, described in details in Section~\ref{sec:scorex}, gives us a possibility to develop an abstract testing framework where it is possible to state contracts for the node view quadruple components without knowing details of their implementations.
% !TEX root = laws.tex

\subsection{Our Contribution}

In this paper we report on the design and implementation of a suite of abstract property tests which are implemented in the Scorex framework to ease checking whether a blockchain client satisfies specified laws. A developer of a concrete blockchain system just needs to implement generators of random test inputs~(such as blocks and transactions), and then the testing system will extensively check properties against multiple input objects. We have implemented 59 property tests. We have integrated the tests into a prototype implementation of the TwinsCoin~\cite{cryptoeprint:2017:232} cryptocurrency, which has two types of blocks. \knote{enhance} 

% !TEX root = laws.tex
\raggedbottom
\subsection{Related Work}

Verification and testing of software systems \cite{myers2011art} is an integral part of a software development lifecycle. Immediately after the implementation of the software, and before its deployment, it has to be verified and tested extensively enough to ensure that all the functional requirements have been properly met. A lot of methods have been developed over the course of time for verification and testing of software. Formal verification \cite{wang2004formal}, for example, is a popular method of program verification~(though testing still remains more prevalent). It is used to validate the correctness of a software module by modeling its behaviour based on a set of formal methods, which are mathematical models specifying the intended functional behaviour. Though there are a lot of formal methods used for this, the most common methods are the ones based on finite state machines \cite{chow1978testing}, Petri Nets, process algebra and timed automata \cite{clarke1996formal}. Prior to formal verification, a mathematical model employing a formal method is decided upon which is then followed by a specification phase where the behaviour of the system is modeled. Finally the actual program is verified against this behaviour specification which is called the verification phase.% In the late 90's, some optimisations were proposed to the otherwise conventional and inefficient exhaustive search methods \mnote{either improve or remove this line}.\cite{holzmann1995improvement}.

Since a software program is developed at module or class level and is integrated with other modules or classes along the development cycle, testing is done at unit level, integration level and system level \cite{myers2011art}, before the software is deployed. End-to-end testing \cite{tsai2001end} is also performed, usually after system testing~(sometimes it is seen as a kind of system testing as well), to validate correct flow spanning different components of the software in real world use cases. Unit tests target individual modules, methods or classes and have a small coverage compared to integration tests which aim towards checking the behaviour of modules when combined together. The two main approaches to unit testing are black box testing and white box testing. The former one focuses on designing test instances without looking inside the code or design, in other words, the black box testing 
is only focusing on the functionality of the unit under testing, while the white box testing approach is more inclined towards testing code coverage i.e. writing test instances which employ the different paths inside the code.

Though initially white box testing was considered as a method suitable for unit testing alone, recently it has emerged as a popular method for integration testing as well. Integration testing is usually done by one or a combination of the following approaches: 
\begin{enumerate}[\IEEEsetlabelwidth{Z}]
\item \textit{Big-Bang approach}.\\In this approach, all the components are integrated together at once and then tested. This method works well for comparatively smaller systems, but is not well suited for larger systems. One obvious disadvantage being that the testing can only begin after all the individual components have been built.
\item \textit{Top-Down approach}.\\As the name suggests, the modules at upper level are tested first and then we move down until we reach the lowest level modules which will be tested at the end. Since lower level modules might not be developed when the upper ones are being tested, stubs are used in place of such the modules. The stubs are trying to simulate behaviour of the modules not implemented yet.
\item \textit{Bottom-Up approach}.\\This approach is opposite to the top-down. Here the lower level modules are tested first and then we iteratively move upwards in the hierarchy until we reach the highest level module. Now as we are testing lower level modules first, stubs are used to simulate the behaviour of higher level modules which may not be implemented yet, if any sibling interaction is required. 
\item \textit{Sandwich approach}.\\ The Sandwich approach is a combination of the Bottom-Up and Top-Down approaches.
\end{enumerate}

Opposed to the conventional unit testing methods which do not take any input parameters, parameterized unit tests~\cite{tillmann2010parameterized} are generalized tests which have an encapsulated collection of test methods whose invocation and behaviour is controlled by a set of input parameters giving more flexibility and automation to unit testing as a whole.

The final full scale testing that a software product undergoes is called the system testing, which includes tests like security test, compatibility test, exceptions handling, scalability tests, stress tests and performance tests. 

In 2013, Visa performed an annual stress test (comes under system testing) to prepare their system VisaNet for the peak traffic of the upcoming holiday season. The test results showed that the system was able to handle 47,000 transactions per second which was around 56\% improvement from the previous year's capacity of the system. %[https://www.visa.com/blogarchives/us/2013/10/10/stress-test-prepares-visanet-for-the-most-wonderful-time-of-the-year/index.html].
Another recent work~\cite{baqer2016stressing} mentions a spam campaign, called \textit{"stress test"}, on the Bitcoin network [Bitcoin cite] caused the network's performance to degrade and essentially resulted in a Denial-of-Service attack, which is cyber-attack on a system where the attack makes the system's resources unavailable or degrades their intended quality to a point where it becomes difficult or sometimes impossible for the honest users to avail the resource. The intention behind this campaign was to expose the vulnerabilities of the network, particularly to spam attacks, and to therefore increase the present transaction verification rate of the network. The authors of the paper~\cite{baqer2016stressing} present an experimental analysis of the above mentioned "attack" by using $k$-means clustering along with some specific features to differentiate between spam and actual transactions. They report that around 23\% of the total transactions flowing on the network were indeed spams during the peak period of the attack. Since this number is a pretty significant number and potentially degrade the network by at least a quarter. As a consequence of this attack, a special type of transactions were observed, called $UTXO$-cleanup transactions, which were created by miners to combine spam transaction together, reducing the $UTXO$ set size, and hence reduce the impact of the attack on the network. 

% ignored text begins
\ignore{ Similar to this, another vulnerability in the Bitcoin system caused MtGox Bitcoin exchange to close in February 2014 [?https://www.businessinsider.in/Bitcoin-Just-Completely-Crashed-As-Major-Exchange-Says-Withdrawals-Remain-Halted/articleshow/30165462.cms]. The exchange announced that close to 850,000 bitcoins were stolen by an attacker who exploited the vulnerability that causes bitcoin transactions to be malleable. Let us denote a bitcoin transaction as the tuple $T = (M, sig$) where $M$ is the message content of transaction and $sig$ is a valid signature on $M$. If a transaction is non-malleable, then it is not viable to construct another transaction $T' = (M, sig')$ such that $sig'$ is also a valid signature on $M$, without the knowledge of the secret key. Due to the fact that in bitcoin a transaction is identified by its unique $id = \mathbb{H}(M, sig)$, where $\mathbb{H}$ is a hash function, and not just $id = \mathbb{H}(M)$ means that $T$ and $T'$ as mentioned above will be treated as different transactions since they will have different $id$, despite the fact that their transaction content is exactly same. The above malleability is possible due to the fact that signature schemes used in bitcoin can be malleable. The way this attack was used to steal money, as claimed by the exchange, is the following:
\begin{enumerate}[\IEEEsetlabelwidth{Z}]
\item A user begins by depositing a certain amount $a$ into exchange's account.
\item He then asks the exchange to transfer his money back to him.
\item The exchange issues a transaction $T$ to transfer $a$ bitcoins to the user's account.
\item The user constructs another transaction $T'$ by exploiting the malleability of transactions.
\item Suppose that somehow $T'$ gets included in the blockchain instead of $T$.
\item This ensures that the user gets $a$ bitcoins in his account. But after this, the user files a request for resending the money claiming that he didn't receive it.
\item To respond to this request, the exchange checks that no transaction with $id = id_T$ ($=\mathbb{H}(T)$) is present and reissues another transaction sending $a$ to the user. This way the user was able to receive double that coins that he were to receive without the vulnerability.
\end{enumerate}
A very intuitive solution [?https://eprint.iacr.org/2013/837.pdf] to this problem is to change bitcoin such that transactions are identified only with $M$ and not the input scripts (signatures). This would mean that even if a signature is forged, the new transaction will hash to the same $id$ as the previous transaction and would eliminate this issue. There is also another solution [?%https://fc15.ifca.ai/preproceedings/bitcoin/paper_9.pdf
] where malleability of bitcoin transaction is dealt-with specific to bitcoin contracts. }
% ignored text ends

These arguments make it pretty convincing that repeated testing of even the most carefully written and designed system is crucial to expose hidden vulnerabilities in the developed system which might miss the eye of the developers. Such tests are performed regularly on important system and help ensure their reliability, security and performance. But, if we talk about Bitcoin and other blockchain system, where the cost of errors can be humungous, then testing becomes the most important part of the software development lifecycle.

In this manuscript we focus on a different method of program testing, called property testing \cite{ron2001property}. This method is concerned with making approximate decisions on whether a function under test satisfies a property globally or not by using only a small number of random input domain elements. We argue that for testing an implementation on top of an abstract framework, like Scorex, this method is of particular interest, since we can split clearly and in a useful way work between the framework and the application. Concretely, properties an application should satisfy are provided before implementing it, and then an application developer just needs to write generators for random input domain elements the tests are using.   
\nocite{holzmann1995improvement}
\nocite{zaki2008formal}

%Some papers to use for references particular to property testing - http://www.wisdom.weizmann.ac.il/~oded/test.html.%
%Integration testing papers for references - https://www.researchgate.net/profile/Xiaoying_Bai/publication/221028427_End-To-End_Integration_Testing_Design/links/02e7e516cabf5c969d000000.pdf%
%Formal verification - http://www.cerc.utexas.edu/~jay/fv_surveys/zaki-AMS-survey-FULL.pdf%
%Formal verification - http://www.cerc.utexas.edu/~jay/fv_surveys/wang_fvsurvey_timed_systems_proc_ieee2004.pdf%
%Unit testing - https://link.springer.com/article/10.1007/s10664-006-5964-9%
%Integration testing - https://link.springer.com/chapter/10.1007/978-3-540-31862-0_18%
%http://citeseerx.ist.psu.edu/viewdoc/download?doi=10.1.1.93.7961&rep=rep1&type=pdf%
%http://ieeexplore.ieee.org/document/1702519/%
%https://dl.acm.org/citation.cfm?id=1767341%




% !TEX root = laws.tex


\subsection{Structure of the Paper}

The paper is organized as follows. In Section~\ref{sec:scorex} we give details on architecture of the Scorex framework. In Section~\ref{sec:props} we describe our approach to property-based testing of blockchain system properties. We present some examples of property tests in Section~\ref{sec:examples}. With Section~\ref{sec:conclusion} we finally conclude.

% !TEX root = laws.tex

\section{Scorex Architecture}
\label{sec:scorex}

Scorex is a framework for implementing blockchain clients. A client is a node in a peer-to-peer network. The node has a local view of the network state. The goal of the whole peer-to-peer system\footnote{concretely, its honest nodes. We skip the notion of adversarial behavior for now.} is to synchronize on a critical part of local views. Scorex splits a node's local view into the following four parts: 

\begin{itemize}
\item{\em history} contains log of persistent modifiers. For example, in a Bitcoin-like blockchain the log is about a sequence of blocks. For an arbitrary persistent modifier, a history implementation can define whether the modifier is valid against it or not.
If the modifier is valid w.r.t history, we call it a {\em syntactically} valid modifier. \knote{explain/example}  
\item{\em minimal state} is a structure enough to check semantics of an arbitrary persistent\bruno{persistent has not been defined in the paper yet} modifier with a constraint that the checking has to be deterministic in nature. If a modifier is valid w.r.t minimal state, we call it a {\em semantically} valid modifier.
\item{\em vault} contains user-specific information. A wallet or additional indexes to get richer information from the blockchain are perfect examples of vault implementation. 
\item{\em memory pool} contains temporary objects to be packed into persistent modifiers. An unconfirmed transaction is a clear example of an object living in the pool.
\end{itemize}

The history and the minimal state are parts of local views to be synchronized across the network. For this, nodes run a consensus protocol to form a proper history, and the history should result in a valid minimal state when persistent modifiers contained in history are applied to a prehistorical state.

The whole node view quadruple is to be updated atomically by applying either a persistent node view modifier or an unconfirmed transaction. Scorex provides guarantees of atomicity and consistency for the application while a developer of a concrete system needs to provide implementations for the abstract parts of the quadruple as well as a family of persistent modifiers.

A central component which holds the quadruple {\em <history, minimal state, vault, memory pool>} and processes its updates atomically is called a {\em node view holder}. The holder is implemented as an actor \knote{link}, so that all received messages are processed in sequence, even if received from multiple threads. If the holder gets a transaction, it updates the vault and the memory pool with it. Otherwise, if the holder gets a persistent modifier, it first updates the history by appending the modifier to it. If appending is successful~(i.e. if the modifier is syntactically valid), the modifier is then applied to the minimal state.  \knote{finish, write about forks}

As an example, we consider the cryptocurrency Twinscoin, which has a hybrid Proof-of-Work and Proof-of-Stake consensus protocol. Scorex has a full-fledged Twinscoin implementation as an example of its usage. Underlying Twinscoin, there are two kinds of persistent modifiers: a Proof-of-Work block and a Proof-of-Stake block. 

% !TEX root = laws.tex

\section{Blockchain properties}

In this section we provide our approach to generalized testing of abstract blockchain-like system design. For extensive testing we are  testing history, minimal state and memory pool separately, and also node view holder which updates the quadruple {\em <history, minimal state, vault, memory pool>} in an atomic way. 

\subsection{Forking}


\knote{forking}

% !TEX root = laws.tex

\section{Examples of properties tests}
In this section, we will provide some examples of property tests which are valid for any blockchain-based system. We have grouped the tests based on their similarity.
\begin{enumerate}[\IEEEsetlabelwidth{Z}]
\item \textit{Tests around Mempool}.\\
Mempool (or Memory Pool) is used to store the unconfirmed transactions which are to be included into persistent modifier which are synonymous to block in a Bitcoin ledger. The following tests are used to check some general properties of a Mempool which every blockchain client should pass.
\begin{itemize}[\IEEEsetlabelwidth{Z}]
\item \textit{A mempool should be able to store a lot of transactions.}\\
In our TwinsCoin implementation, we have tested that the mempool stores at least 1000 transactions.
\item \textit{Filtering of valid and invalid transactions from a mempool should be fast}.\\
We checked our implementation of mempool to filter invalid transactions in the time that it takes to evaluate 500 dummy blocks of SHA-256 hash.
\item \textit{Transactions successfully added to memory pool should be available by id.}\\
The purpose of this test is to ensure that once a transaction is added to the mempool, it indeed is available by id meaning that the transaction can be queried from the mempool implying that the mempool has added the transaction successfully in its data structure.
\end{itemize}
\item \textit{Tests around Persistent Modifiers}.\\
A modifier is the building block of a blockchain and is used to update history and the minimal state of the system. As soon as a valid modifier gets appended to history, a node's view changes in the sense that the history is updated along with the minimal state.
\begin{itemize}[\IEEEsetlabelwidth{Z}]
\item \textit{Valid persistent modifier should be successfully applied to history and available by id after that}.\\
A Totally valid persistent modifier once applied to history should be available by a unique id after that which can be queried from the history data structure. The importance of this test comes from the fact that it is of utmost importance for the client implementation to tell the difference between the modifiers that have been appended to the history from those that haven't been added. For this purpose, the unique id of the modifier can be used to query the history to know if the modifier has been added to the history of not.
\item \textit{Invalid modifiers should not be able to be added to history.}\\
An invalid modifier (can be syntactically or semantically invalid) should not be added to history and hence, should not be available by id.
\item \textit{Application of invalid modifier (inconsistent with the previous ones) should be unsuccessful}.\\
When the unique id of an invalid modifier is queried from history, it should return NULL (always) showing that the invalid modifier was not added to the history.
\end{itemize}

\end{enumerate}



%- Valid box should be successfully applied to state, it's available by id after that.
%- State should be able to generate changes from valid block and apply them.
%- Wallet should contain secrets for all it's public propositions.
%- State changes application and rollback should lead to the same state and the component changes should also be rolled back.
%- Transactions once added to a block should be removed from the local copy of mempool.
%- Minimal state should be able to add and remove boxes based on received transaction's validity.
%- Modifier (to change state) application should lead to new minimal state whose elements' intersection with previous ones is not complete (at least some new boxes are introduced and some previous ones removed).
%- Application of the same modifier twice should be unsuccessful.
%- Application of a valid modifier after rollback should be successful.
%- Once an invalid modifier is appended to history, then history should not contain it and neither should it be available in history by it's id.
%- History should contain valid modifier and report if a modifier if semantically valid after successfully appending it to history.
%- BlockchainSanity test that combines all this test.
% !TEX root = laws.tex

\section{Conclusion}
\label{sec:conclusion}

For generic abstract modular Scorex framework, we have implemented a suite of property-based tests.

\section*{Acknowledgments}

Authors would like to thank Bruno Woltzenlogel Paleo for his valuable comments and corrections.

\bibliographystyle{IEEEtran}
\bibliography{sources, ref}

%Appendices

%\newpage
%\appendix
%% !TEX root = laws.tex

\section{Tests Implemented}





\end{document}
