% !TEX root = laws.tex

\section{Scorex Architecture}

Scorex is a framework for implementing blockchain clients. A client is a node in a peer-to-peer network. The node has a local view of the network state. The goal of the whole peer-to-peer system~\footnote{concretely, honest nodes of it. We skip a notion of adversarial behavior further.} is to synchronize on a critical part of local views. Scorex is breaking node's local view into the following four parts: 

\begin{itemize}
\item{\em history} contains log of persistent modifiers. AFor example, in a Bitcoin-like blockchain the log is about a sequence of blocks. For an arbitrary persistent modifier, a history implementation can define whether the modifier is valid against it or not.
If the modifier is valid, we call it the {\em syntactically} valid modifier. \knote{explain/example}  
\item{\em minimal state} is a structure enough to check semantics of an arbitrary persistent modifier
\item{\em vault} contains user-specific information. A wallet or additional indexes to get richer information from the blockchain are perfect examples of vault implementation. 
\item{\em memory pool} contains temporary objects to be packed into persistent modifiers. An unconfirmed transaction is a clear example of an object living in the pool.
\end{itemize}

The history and the minimal state are the parts of local views to be synchronized across the network. For this, nodes are running some consensus protocol to form a proper history, and the history should be resulted in a valid minimal state which is got by application of history elements to a prehistorical state.

The whole node view quadruple is to be changed atomically by applying whether a persistent node view modifier or an unconfirmed transaction. Scorex provides guarantees of atomicity and consistency for the application while a developer of a concrete system needs to provide implementations for the abstract parts of the quadruple as well as a family of persistent modifiers.

A central component which is holding the quadruple {\em <history, minimal state, vault, memory pool>} and processing its updates atomically is called {\em node view holder}. The holder is implemented as an actor \knote{link}, so it is processing all the messages it is getting in sequence even if being requested from multiple threads. If the holder is getting a transaction, it updates the vault and the memory pool with it. Otherwise, if the holder is getting a persistent modifier, it is first updating the history by appending the modifier to it. If appending is successful~(we say in this case that the modifier is syntactically valid). 

As an example, we consider a cryptocurrency Twinscoin, which consensus protocol is a combination of Proof-of-Work and Proof-of-Stake. Scorex has a full-fledged Twinscoin implementation as an example of its usage. In Twinscoin, there are two kinds of persistent modifiers, a Proof-of-Work block, and a Proof-of-Stake block. 
